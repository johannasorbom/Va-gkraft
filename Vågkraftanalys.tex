\documentclass[10pt,a4paper,oneside]{article}

\usepackage[swedish]{babel}
\usepackage[utf8]{inputenc}
\usepackage[T1]{fontenc}
\usepackage{graphicx}
\usepackage{hyperref}
\usepackage{url}
\usepackage[nomarkers]{endfloat}
\renewcommand{\efloatseparator}{\mbox{}} 
\mdseries\itshape\urlstyle{same}
 
\title{Vågkraft \\ 
\large En problemanalys}
\author{\small Johanna Sörbom}
\date{\small \today}

\begin{document}

\maketitle
\newpage

\section{Sammanfattning}
\newpage

\tableofcontents
\newpage

\section{Inledning}
Vindkraft är en energiform som har användas i många olika former under under människans historia (2). Så tidigt som 5000 år före kristus utnyttjades vindens kraft för att driva båtar längs med Nilen och väderkvarnar användes för pumpa vatten i Kina. Sedan dess har nya sätt att utnyttja vindens energi utvecklats och spridits runt om i världen (3). Vågkraft är ett relativt nytt sätt att utnyttja vindens energi. När vind rör sig över vatten skapas vågor genom friktionskraft som lagrar vindens energi under en tidsperiod (2). Om all denna energi utvanns så skulle det räcka för att täcka hela jordens elkonsumtion (1). Så varför gör den inte redan det? Denna rapport kommer att presentera en bild av dagens vågkraft samt analysera de problem som står i vägen för en storskalig vågkraftsutvinning i världen. 
\newpage

\section{Metod}
Denna rapport grundar sig på litteraturstudier från tidigare rapporter innom ämnet vågkraft. 

\section{Källkritik}

\section{Bakgrund}
Vad är vågkraft? 

Varför vill man utvinna vågkraft?
Statistik om energi 
Möjligheter 

Hur ser dagens vågkraft ut?
Exempel runt om i världen
Hur mycket satsas på vågkraft?
Hur utbrett är det.

Vad är problemen med dagens vågkraft (reverse salients)?
Ekonomi
Effektivitet 
m.m 

\section{Diskussion}

\section{Slutsats}

\section{Källor}
1. \url{https://link-springer-com.focus.lib.kth.se/content/pdf/10.1007%2F978-3-540-74895-3.pdf} \\
2. \url{http://iopscience.iop.org.focus.lib.kth.se/book/978-0-750-31040-6.pdf} \\
3. \url{http://windenergyfoundation.org/about-wind-energy/history/} \\

\bibliographystyle{amsplain}
\bibliography{Vågkraftlitterature}

\end{document}